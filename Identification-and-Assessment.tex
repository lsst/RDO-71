\section{Risk \& Opportunity Identification and Assessment}
\label{sec:id-assess}

This section describes who and how risks and opportunities are identified and assessed.
This is currently written assuming that the Risk Register is implemented in NOIRLab's ATS Risk Tool.

\subsection{Create Risk Item}

Minimum content before saving in the Risk Register:
\begin{itemize}
	\item Project --- Rubin Operations.
	\item Risk Type --- ``Threat'' or ``Opportunity''.
	\item Risk Department --- Select Rubin Observatory department which owns the risk and its assessment.
	\item Risk Owner --- Point-of-contact for ownership, technical information, and responsible for orchestrating the response to the risk.
	\item Risk Category --- Select from list provided by Section \ref{sec:risk-categories}.
	\item Risk Title --- Short, descriptive title.
	\item Risk Statement --- Description of risk, which must include the IF-THEN statement at a minimum.
	\item Status --- Automatically set as ``Candidate``.
\end{itemize}

Optional fields:
\begin{itemize}
	\item Risk Sub Category --- Select from list provided by Section \ref{sec:risk-categories}.
	\item Parent --- Select any applicable Parent risk(s).
\end{itemize}

Risks can be created as ``children'' of ``parent'' risks.
This is appropriate for a risk that is anticipated as being realized in separate time-ordered phases, or as distinct parallel events that each have their own impact.
The cost and schedule impact of the parent is the combined cost and schedule impact of its children (see Section \ref{sec:risk-review} for details.
In order for the combined impact to be simple to understand and estimate, child risks must be constructed to be fully independent (non-overlapping) and a complete set.
(Risk owners who are considering breaking down a risk into component children should only make that move if they can satisfy these criteria.)

\subsection{RROB Candidate Risk Review}
\label{sec:risk-review}

Risks are reviewed by the Risk Owner before submitting for review by the RROB.
The Risk Owner is responsible for verifying assessment in the automatically generated fields in the Risk Register.
The RROB reviews for sufficiency and accuracy of information and confirms ability of Risk Owner to manage effectively as an Active risk.

Minimum content before submitting a Candidate Risk for review by the RROB:
\begin{itemize}
	\item Status Description --- Records a list of Rubin teams involved with defining, assessing or implementing risk and plans.
	\item Existential Risk --- The RROB will confirm or deny if a risk is existential.
	\item Cost Impact --- Select from list provided by Table \ref{table:risk-cost-impacts}.
	\item Schedule Impact --- Select from the list provided by Table \ref{table:risk-schedule-impacts}.
	\item Likelihood --- Select from the list provided by Table \ref{table:risk-likelihood}.
	\item Residual Likelihood --- Select from the list provided by Table \ref{table:risk-likelihood}.
	\item Cost and/or Schedule Impact Description --- Narrative describing analysis and impact assessment.
	\item Cost and/or Schedule fields under Analyze Risk Quantitative, if impact is non-negligible. --- Quantitative analysis and impact assessment.
		\begin{itemize}
			\item Likely Cost --- The estimated post-mitigation cost of addressing the risk in the year after it is realized.
			(All these short-term cost estimates are used to forecast the required annual level of reserve funds.)
			\item Likely Delay --- The estimated length of delay to the ``Impacted Milestone.''
			\item Impact Milestone --- Event related to the realization of a risk.
			If the ``Impacted Milestone'' is either ``Start of Operations'' or ``LSST Survey Completion,'' the ``Realized Risk Plan'' (a separate item in this list) needs to include ``Extend Survey'' because such schedule delays incur a post-operations cost.
		\end{itemize}
	\item Plan Type, and a minimum of one related Response Plan (i.e., how the risk will be mitigated). --- Select from the plan type list provided by Section \ref{sec:response-types}.
	\item Realized Risk Plan --- Description of event trigger, and the plan for how the risk will be addressed once it is realized.
\end{itemize}

Optional fields:
\begin{itemize}
	\item Overall Impact --- Impact category that can override the automatically generated cost and schedule impact categories.
	Note that the ``Impact Severity'' will be set to the most severe of the three impact categories.
	\item Related Actions
\end{itemize}

When analyzing a risk that is the parent of $N$ child risks:

\begin{itemize}
	\item First analyze the children separately, i.e., estimating Residual Probability, $P_k$, Likely Cost, $C_k$, and Likely Delay, $S_k$, for the $k$-th risk, repeating for all $k$ in the range $(1 \ldots N)$.
	\item For each child risk, calculate its Cost Exposure: $E_k = C_k \times P_k$.
	\item The combined Cost Exposure for the parent risk is the sum of the Cost Exposures of its children: $E_{\rm total} = \sum_{k=1}^N E_k$.
	\item The Likely Cost of the parent risk is the sum of the Likely Costs of its children: $C_{\rm total} = \sum_{k=1}^N C_k$.
	\item The Residual Probability of the parent risk is its Cost Exposure divided by its Likely Cost: $P_{\rm total} = E_{\rm total} / C_{\rm total}$.
	This is the cost-weighted average of the child risks' Residual Probabilities, and it is the most meaningful definition of probability for parent risks.
	\item Other quantitative attributes of parent risks can be computed by simple summation (Minimum Cost, Maximum Delay, Schedule Exposure, etc.).
	\item Pre-response quantities of parent risks, such as Impact Scores, can be reverse engineered from their Residual Probabilities.
\end{itemize}

\subsection{Continuous Review and Updates to Active Risks}

Risk Owners flag issues or proposed changes to Active Risks for review at an RROB monthly meeting.

Managing departments should review risks on a frequency based on severity.

In the (May) Annual Scrub, Level 3 Team Leaders are invited to review, in the ``Scrub Sandbox,'' pertinent fields of risks that their team is affected by and which they are expected to help respond to.
Modifications they propose are reviewed by the RROB and implemented in the Risk Register by the Risk Owners in the post-scrub implementation phase (June).

\subsection{Addressing Realized Risks}

Triggering a Realized Risk:
Risk Owner reviews all Response Plans and Actions, and there is a follow-up review with department management.
Risk Owner proposes change of Status from ``Active'' to ``Realized.''
Review by the RROB is required if the severity is above a threshold determined by RROB Chair.

Addressing a Realized Risk:
Risk Owner works with department management to submit a Request Beyond Target (RBT) for additional resources needed to address the risk (see \citeds{RTN-005} for the RBT process).
RBT is reviewed by RDO, which works with department for any needed escalation to NOIRLab, SLAC, funding agencies, or Resource Forum.

Additional resources for addressing risks are drawn from reserve funds.
The annual budget for the reserve is estimated as the sum of the cost exposure over all parent risks.
Currently, the cost exposure and budget calculations are performed outside the Risk Register in a companion tool (the ``Rubin View of the NOIRLab Risk Register'' Google sheet).

While a risk is in the ``Realized'' state, the (on-going) quantitative analysis should yield the Residual Probability and associated Likely Cost and Likely Delay appropriate for the risk after both mitigation and partial (to-date) addressing of the risk.
(Some risks take significant time, and multiple rounds of resource planning, to address.)
This ensures that the Cost Exposure associated with the risk only captures the funds that are still likely to be required, not the total needed to fully address the risk.

\subsection{Retiring Risks}

Risk Owner reviews the risk prior to close-out, be it a realized risk or if the risk's event cannot trigger.
The RROB reviews the risk's associated plans and actions before retiring risk that did not trigger to ensure there is no impact to other risks.
All actions should be completed in the Risk Register (could include Jira tickets for follow-up work) before retiring risk that did trigger.
Documentation updates should be considered during this review.
A conclusion statement is included in the Risk Register to record the reason for retiring or depreciating a risk.

\subsection{Risk Assessment}

There are six aspects to assessing the state of each risk and opportunity within a subsystem:
\begin{enumerate}
\item Identification: identifying elements of risk in the department’s activity.
\item Establishing time frame: determining the likely time at which an event would come to pass.
\item Assessing probability: estimating the probability that an undesirable event may occur.
\item Assessing severity: gauging the severity of the impact that such an event would have on the status of the project if the event were to occur.
\item Developing risk options: developing plans to avoid, accept, mitigate, or transfer.
\item Developing a management response: consider how the project may respond if the event should occur.
\end{enumerate}
