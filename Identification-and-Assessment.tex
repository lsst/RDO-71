\section{Risk \& Opportunity Identification and Assessment}
\label{sec:id-assess}

This section describes who and how risks and opportunities are identified and assessed.
This is currently written assuming that the Risk Register is implemented in NOIRLab's ATS Risk Tool.

\begin{enumerate}
	\item The managing group creates a risk or opportunity in the Risk Register as a "Candidate."
	\item RROB reviews content, then either asks for more information or sets risk to "Active."
	\item After
\end{enumerate}

\subsection{Create Risk Item}

Minimum content before saving in the Risk Register:
\begin{itemize}
	\item Project --- Rubin Operations
	\item Risk Type --- Threat or Opportunity
	\item Risk Department --- Select Rubin Observatory department which owns the risk and its assessment
	\item Risk Owner --- Point-of-contact for ownership and technical information regarding risk. Responsible for orchestrating the response to the risk.
	\item Risk Category --- Select from list (need to add list to RDO-71)
	\item Risk Title --- Short, descriptive title
	\item Risk Statement --- Description of risk, which must include the IF-THEN statement at a minimum.
	\item Status --- Automatically set as ``Candidate''
\end{itemize}

Optional fields:
\begin{itemize}
	\item Risk Sub Category --- Select from list (need to add list to RDO-71)
	\item Parent --- Any applicable Parent risks
\end{itemize}

Risks can be created as ``children'' of ``parent'' risks.
This is appropriate for a risk that is anticipated as being realized in separate time-ordered phases, or as distinct parallel events that each have their own impact.
The cost and schedule impact of the parent is the combined cost and schedule impact of its children, see the next section for details.

\subsection{RROB Candidate Risk Review}

Should be reviewed by the Risk Owner before submitting.
RROB to review for sufficiency and accuracy of information, to confirm ability of Risk Owner to manage effectively as an Active risk.

Minimum content before submitting a Candidate Risk for review by the RROB:
\begin{itemize}
	\item Cost Impact --- According to risk table (need to add risk table to RDO-71).
	\item Likelihood --- According to risk table (need to add risk table to RDO-71)
	\item Schedule Impact --- According to risk table (need to add risk table to RDO-71)
	\item Cost and/or Schedule Impact Description
	% if over certain severity. PJM: I think we always want a brief description of any impact we specify.
	\item Cost and/or Schedule fields under Analyze Risk Quantitative, if impact is non-negligible.
	The Likely Cost is the estimated cost of addressing the risk in the year after it is realized.
	(All these short-term cost estimates are used to forecast the required annual level of reserve funds.)
	The Likely Delay is the estimated length of delay to the Impacted Milestone.
	If the Impacted Milestone is either ``Start of Operations'' or ``LSST Survey Completion,'' the Realized Risk Plan (see below) needs to include ``Extend Survey.''
	(Such schedule delays incur a post-ops cost.)
	\item Plan Type, and a minimum of one related Response Plan (i.e. how the risk will be mitigated)
	\item Realized Risk Plan --- Description of event trigger, and the plan for how the risk will be addressed once it is realized.
	\item All fields under Analyze Risk Quantitative --- note that these do not automatically populate any required fields.
	\item Existential Risk
\end{itemize}

Optional fields:
\begin{itemize}
	\item Overall Impact --- Impact category that can override the automatically generated cost and schedule impact categories (sets to most severe)
	% \item Cost or Schedule Impact Description. PJM: See above, I think we always need a description any time we specify an impact.
	\item Related Actions
\end{itemize}

Verify assessment in the automatically generated fields under Analyze Risk Score.

When analyzing a risk that is the parent of $N$ child risks:
\begin{itemize}
	\item First analyze the children separately, i.e. estimating Residual Probability $P_k$, Likely Cost $C_k$, and Likely Delay $S_k$ for the $k$-th risk, and repeating for all $k$ in the range $(1 \ldots N)$.
	\item For each risk, calculate the Cost Exposure $E_k = C_k \times P_k$.
	\item The combined Cost Exposure for the parent risk is just the sum of the Cost Exposures of its children, $E_{\rm total} = \sum_{k=1}^N E_k$.
	\item The Likely Cost of the parent risk is the sum of the Likely Costs of its children, $C_{\rm total} = \sum_{k=1}^N C_k$.
	\item The Residual Probability of the parent risk is its Cost Exposure divided by its Likely Cost, $P_{\rm total} = E_{\rm total} / C_{\rm total}$.
	This is just the cost-weighted average of the child risks' Residual Probabilities, and is the most meaningful definition of probability for parent risks.
	\item Other quantitative attributes of parent risks can be computed by simple summation (e.g. Minimum Cost, Maximum Delay, Schedule Exposure, etc).
	\item Pre-response quantities of parent risks, such as Impact Scores, can be reverse engineered from their Residual Probabilities.
\end{itemize}

\subsection{Continuous Review and Updates to Active Risks}

Risk Owners flag issues or proposed changes to Active Risks for review at an RROB monthly meeting.

Managing departments should review risks on a frequency based on severity.

In the (May) Annual Scrub, Level 3 Team Leaders are invited to review, in the Scrub Sandbox, pertinent fields of risks that their team is affected by and which they are expected to help respond to.
Modifications they propose are reviewed by the RROB and implemented in the Risk Register by the Risk Owners in the post-scrub implementation phase (June).

\subsection{Addressing Realized Risks}

Triggering a Realized Risk:
Review of all Response Plans and Actions by Risk Owner, and follow-up review with department management.
Risk Owner proposes change of Status from Active to Realized.
Review with RROB is required if severity is above threshold (determined by RROB Chair).

Addressing a Realized Risk:
Risk Owner works with department management to submit a Request Beyond Target (RBT) for additional resources needed to address the risk (see RTN-005 for the RBT process).
RBT is reviewed by RDO, which works with department for any needed escalation to NOIRLab, SLAC or the funding agencies or Resource Forum.

Additional resources for addressing risks are drawn from reserve funds.
The annual budget for the reserve is estimated as the sum of the cost exposure over all parent risks.
Currently, the cost exposure and budget calculations are performed outside the Risk Register in a companion tool (the ``Rubin View of the NOIRLab Risk Register'').

\subsection{Retiring Risks}

Review of close-out due to realized risk, or if event cannot trigger.
RROB to review before retiring risk that didn't trigger.
All actions should be completed in the Risk Register (could include Jira tickets for follow-up work) before retiring risk that did trigger.
Documentation updates should be considered.

\subsection{Risk Assessment}

There are six aspects to assessing the state of each risk and opportunity within a subsystem:
\begin{enumerate}
\item Identification:  identifying elements of risk in the department’s activity.
\item Establishing time frame:  determining the likely time at which an event would come to pass.
\item Assessing probability:  estimating the probability that an undesirable event may occur.
\item Assessing severity:  gauging the severity of the impact that such an event would have on the status of the project if the event were to occur.
\item Developing risk options:  developing plans to avoid, accept, mitigate, or transfer.
\item Developing a management response: consider how the project may respond if the event should occur.
\end{enumerate}
