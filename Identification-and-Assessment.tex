\section{Risk \& Opportunity Identification and Assessment}
\label{sec:id-assess}

This section describes who and how risks and opportunities are identified and assessed.


\begin{enumerate}
	\item The managing group creates a risk or opportunity in the ATS Risk Tool as a "Candidate."
	\item RROB reviews content, then either asks for more information or sets risk to "Active."
	\item After 
\end{enumerate}

\subsection{Create Risk Item}

Minimum content before saving in Risk Tool:
\begin{itemize}
	\item Project --- Rubin Operations
	\item Risk Type --- Threat or Opportunity
	\item Risk Department --- Select Rubin Observatory department which owns the risk and its assessment
	\item Risk Owner --- Point-of-contact for ownership and technical information regarding risk
	\item Risk Category --- Select from list (need to add list to RDO-71)
	\item Risk Title --- Short, descriptive title
	\item Risk Statement --- Description of risk, which must include the IF-THEN statement at a minimum.
\end{itemize}

Optional fields:
\begin{itemize}
	\item Risk Sub Category --- Select from list (need to add list to RDO-71)
	\item Parent --- Any applicable Parent risks
\end{itemize}

Automatically generated fields:
\begin{itemize}
	\item Risk ID --- Automatically generated unique identifier
	\item Status --- Automatically set as "Candidate"
\end{itemize}

\subsection{RROB Candidate Risk Review}

Should be reviewed by POC before submitting.
RROB to review for sufficiency and accuracy of information for managing group for ability to manage effectively as an Active risk.

Minimum content before submitting to RROB:
\begin{itemize}
	\item Cost Impact --- According to risk table (need to add risk table to RDO-71)
	\item Likelihood --- According to risk table (need to add risk table to RDO-71)
	\item Schedule Impact --- According to risk table (need to add risk table to RDO-71)
	\item Cost or Schedule Impact Description, if over certain severity.
	\item Cost or Schedule fields under Analyze Risk Quantitative, if over certain severity.
	\item Plan Type, and a minimum of one related Reponse Plan
	\item Realized Risk Plan --- Description of event trigger
\end{itemize}

Optional fields:
\begin{itemize}
	\item Overall Impact --- Impact category that can override the automatically generated cost and schedule impact categories (sets to most severe)
	\item Cost or Schedule Impact Description
	\item All fields under Analyze Risk Quantitative --- note that these do not automatically populate any required fields.
	\item Existential Risk --- RROB or POC set?
	\item Related Actions
\end{itemize}

Automatically generated fields:
\begin{itemize}
	\item All fields under Analyze Risk Score.
\end{itemize}

\subsection{Continuous Reivew and Updates to Active Risks}

TBD for RROB monthly meetings.
Managing departments should review risks on a frequency based on severity.

\subsection{Realized Risk Trigger}

Reivew of all Reponse Plans and Actions by POC, and follow-up review with department management.
Department management review with RROB (required if over certain severity).
Review with DO, RROB and department for any escalation, with any review and discussion with NOIRLab.

\subsection{Retiring Risks}

Review of close-out due to realized risk, or if event cannot trigger.
RROB to review before retiring risk that didn't trigger.
All actions should be completed in the Risk Tool (could include Jira tickets for follow-up work) before retiring risk that did trigger.
Documentation updates should be considered.
