\section{Terminology: Risk \& Opportunity Response\, Monitoring\, and Control}

This section defines terminology relating to how risks and opportunities are responded to.

\subsection {Risk \& Opportunity Response}

The definition of responses are included in the ATS Risk Software.
The following include the definition of response plans to risks, and additional comments are included in \href{RTN-051.lsst.io}.

\begin{itemize}
	\item \textbf{Avoid} ---
	changing your strategy or plans to avoid the risk. This risk response strategy is about removing the threat by any means. That can mean changing your management plan to avoid the risk because it’s detrimental to the project/program.

	\item \textbf{Transfer} ---
	passing ownership and/or liability to a third party - Transfer or pass the work on resolving the risk to a third party. For e.g. purchase fire insurance for an unfinished building.

	\item \textbf{Mitigate} ---
	reducing the probability and/or impact of the risk below a threshold of acceptability - Some risks cannot be avoided and need to take action to reduce the impact of the risk. For e.g. work procedures and equipment designed to reduce workplace safety risks.

	\item \textbf{Accept} ---
	recognizing residual risks and devising responses to control and monitor them - This risk response strategy consists of identifying a risk and documenting all the risk management information about it, but not taking any action unless the risk is realized.
\end{itemize}

The following include the definition of response plans to opportunities.

\begin{itemize}
	\item \textbf{Exploit} ---
	Exploiting a risk to make use of the opportunity that becomes available if that risk occurs.

	\item \textbf{Share} ---
	Distributing the risk across multiple stakeholders (teams/projects/programs)

	\item \textbf{Enhance} ---
	Enhancements is an action that is taken to increase the chance of the opportunity occurring.

	\item \textbf{Ignore} ---
	Opportunities that cannot be actively addressed through exploiting, sharing or enhancing can perhaps be ignored, with no special measures being taken to address them.
\end{itemize}
