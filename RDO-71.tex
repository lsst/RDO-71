\documentclass[DM,toc,lsstdraft]{lsstdoc}
% lsstdoc documentation: https://lsst-texmf.lsst.io/lsstdoc.html

% Generated by Makefile
\input{meta}

% Package imports go here.

% Local commands go here.

% If you want glossaries, uncomment:
% \input{aglossary.tex}
% \makeglossaries

\title{Rubin Observatory Risk and Opportunity Management Plan}
% \setDocSubtitle{Optional subtitle}

\author{%
Austin Roberts,
Robert Blum,
Chuck Claver,
Leanne Guy,
Phil Marshall,
Wil O'Mullane,
Kevin Siruno,
Matthew Rumore.
}

\setDocRef{RDO-71}
\setDocUpstreamLocation{\url{https://github.com/lsst/RDO-71}}
\date{\vcsDate}
\setDocCurator{TBD}

\setDocAbstract{%
The document provides the Rubin Observatory Risk and Opportunity Management Plan.
}

% Revision history.
% Order: oldest first.
% Fields: VERSION, DATE, DESCRIPTION, OWNER NAME.
% See LPM-51 for version number policy.
\setDocChangeRecord{%
  \addtohist{1}{2022-12-12}{Unreleased.}{Matthew Rumore}
}

\begin{document}

\maketitle



\section{Introduction and Background}

This document describes the Risk \& Opportunity Management Plan used to identify, assess, respond to, and manage risks and opportunities associated with the technical, cost and schedule aspects for the Vera C.\ Rubin Observatory throughout the operation of the Legacy Survey of Space and Time (LSST).
The Rubin Observatory Risk \& Opportunity Management Plan recognizes the benefits of managing uncertainty during operations in a holistic and systematic manner.

The Risk \& Opportunity Management Plan is one plan in a group of Rubin Observatory Operations plans that collectively define a comprehensive approach to managing risk, and opportunity, which are collectively known as approaches to making decisions under conditions of uncertainty.
Risk of harm to personnel and equipment is the focus of the Rubin Observatory Safety Policy, \citeds{RDO-15}.
Further aspects of risk and safety management are codified under the NSF (National Science Foundation) NOIRLab and the Department of Energy (DOE) SLAC Laboratory (SLAC National Accelerator Laboratory) documents where appropriate. 
Additionally and were applicable, risk, safety, and hazard analysis plans are adopted for Rubin Observatory Operations from the Rubin Observatory Construction Project.

The key aspects of the Risk \& Opportunity Management Plan are:
\begin{itemize}
	\item A standard methodology to identify and assess major risks and opportunities associated with Operations work breakdown structure (WBS) elements and operations functions from the Operations Plan.

	\item A continuous process to review and re-assess current risks and opportunities on a quarterly or semi-annual basis and address new risks and opportunities as they emerge.
	
	\item Common techniques for assignment of budget and schedule for the anticipated response in the event of a realized risk.
	
	\item An approach and tool to measure and compare to contingency levels, the remaining major risk exposure across operation.
	
	\item A single dynamic and interactive system to inform management, to support communication across operations, to facilitate and encourage regular participation of team members, and to produce standard reporting and tracking features.
\end{itemize}


\subsection{Risk Management Process Overview}

The risk management process is a continuous and proactive approach to keeping risk at an acceptable level through awareness, tracking, and response handling.
The Rubin Observatory Risk Management process is an event-centric approach.
It is characterized by the identification of events that may occur in the future with resulting negative or positive consequences for operations.
There are different types of risk associated with Rubin Observatory Operations:
\begin{itemize}
\item Technical Risk, consisting of the risk of not meeting survey performance requirements or deliverables;
\item Cost Risk, consisting of the risk that the available budget will be insufficient to cover the scope of operations;
\item Schedule Risk, consisting of the risk that the survey will fail to meet scheduled milestones; and,
\item a related category of risk is called Programmatic Risk, which is risk produced by events that are beyond the control of the operations management team, where programmatic Risk can be a source of risk in any of the other three risk categories. 
\end{itemize}

In transitioning from the construction phase to operations, the operations team will consult with the Rubin Observatory project to understand any project risks that might transfer to operations or (more likely) evolve into operations risks.
During the pre-operations phase, which occurs in parallel with the project integration and commissioning phases, the operations team will work closely with the project team to understand all risks, safety plans, and hazards and adapt, revise, and establish analogous processes, procedures, and policies.
This transition will take place from the beginning of pre-operations and be completed no later than the Operations Readiness Review (ORR), which marks formal handover of the Rubin Observatory system from the project to the operations team.

\subsection{Risk Management Tools}

Rubin Observatory Operations will follow the NOIRLab model, shown Figure \ref{fig:NOIRLab-risk-model} for managing its risks and opportunities. 
The Alcea Tracking Solutions (ATS) software tool for risk management, which has been adopted by NOIRLab, will used by Rubin.
The user guide for the ATS tool is found in \citeds{RTN-051}.

In addition to this, Rubin may choose to implement additional tools on top of the Alcea tool to better integrate the process of risk and opportunity management into the Rubin workflow. 
Any additional such tools developed by Rubin for managing risks will be added to this document as they are developed. 
 
All risks are reported to AURA and SLAC management, as well as NSF through NOIRLab quarterly reports.

\begin{figure}[t]
\caption{NOIRLab Risk Management Model.}
\centering
\includegraphics[width=\textwidth]{NOIRLab-risk-model-temp}
\label{fig:NOIRLab-risk-model}
\end{figure}


\subsection{Roles and Responsibilities}

The Rubin Observatory Risk \& Opportunity Board (RROB) serves as the managing group for the Risk \& Opportunity Management Plan.
The RROB is managed by the System Performance (SP) Department.

The \textbf{Rubin Observatory Operations Director} has overall responsibility for managing and controlling operations risks.
The Director will work with the senior managers to review and assess current risks on a quarterly basis.
The Director will also approve all new risks (or delegate this responsibility as appropriate to another operations manager) in coordination with the RROB.

The \textbf{Rubin Observatory Executive Council} is the group of operations management and leadership staff charged with reviewing the risk registry, evaluating the risk and opportunity assessments, collaborating on risk handling options, and developing implementation recommendations, which are forwarded to the Director of operations.
The Executive Council consists of the Associate Directors (ADs) and Deputy Directors as well as other operations staff.
It is expected that the head of safety for NOIRLab will meet with the Executive Council to regularly review risks.


\section{Terminology: Risk \& Opportunity Management Process}
\label{sec:process}

This section defines terminlogy used in describing the management of risks and opportunities at Rubin.
Risks and opportunities are set to have one of the following Risk Register statuses and follow the lifecycle described in Figure \ref{fig:risk-lifecycle}.

\begin{itemize}
	\item \textbf{Candidate} ---
	Risks and opportunities in a draft state, which are not actively managed by the project.

	\item \textbf{Active} ---
	Risks and opportunities deemed valid, and actively managed by the project.

	\item \textbf{Realized} ---
	Risks and opportunities which have been realized.
	There are three models available for the trigger:
		\begin{itemize}
			\item \textbf{Specific Trigger Date} ---
			A specific calendar date when contingency funds must either be obligated to respond to a risk, the risk can be retired, or an opportunity's beneficial event will occur.

			\item \textbf{Random Occurrence} ---
			Certain risks or opportunities are known to potentially occur but their date(s) are random; for example, critical staff may depart the project, weather delay, equipment failure.
			This type of event requires an estimate of the number of random occurrences and the cost of each.

			\item \textbf{Distributed Occurrence} ---
			Identical risks or opportunities are sometimes distributed throughout periods in the project; for example, software packages are evaluated for performance on an annual basis.
			This type of event distributes the possible contingency obligation profile over the specified time span.
		\end{itemize}

	\item \textbf{Retired} ---
	Risks and opportunities which can no longer valid or actively managed, as they have been realized or the event trigger can no longer occur.

	\item \textbf{Deprecated} ---
	Risks and opportunities that were deemed invalid and are were not actively managed.
\end{itemize}

\section{Diagram: Risk \& Opportunity Management Process}

\begin{figure}[t]
\caption{Lifecycle of Risks.}
\centering
\includegraphics[width=\textwidth]{risk-lifecycle-temp}
\label{fig:risk-lifecycle}
\end{figure}


\section{Risk \& Opportunity Identification and Assessment}
\label{sec:id-assess}

This section describes who and how risks and opportunities are identified and assessed.


\begin{enumerate}
	\item The managing group creates a risk or opportunity in the ATS Risk Tool as a "Candidate."
	\item RROB reviews content, then either asks for more information or sets risk to "Active."
	\item After 
\end{enumerate}

\subsection{Create Risk Item}

Minimum content before saving in Risk Tool:
\begin{itemize}
	\item Project --- Rubin Operations
	\item Risk Type --- Threat or Opportunity
	\item Risk Department --- Select Rubin Observatory department which owns the risk and its assessment
	\item Risk Owner --- Point-of-contact for ownership and technical information regarding risk
	\item Risk Category --- Select from list (need to add list to RDO-71)
	\item Risk Title --- Short, descriptive title
	\item Risk Statement --- Description of risk, which must include the IF-THEN statement at a minimum.
\end{itemize}

Optional fields:
\begin{itemize}
	\item Risk Sub Category --- Select from list (need to add list to RDO-71)
	\item Parent --- Any applicable Parent risks
\end{itemize}

Automatically generated fields:
\begin{itemize}
	\item Risk ID --- Automatically generated unique identifier
	\item Status --- Automatically set as "Candidate"
\end{itemize}

\subsection{RROB Candidate Risk Review}

Should be reviewed by POC before submitting.
RROB to review for sufficiency and accuracy of information for managing group for ability to manage effectively as an Active risk.

Minimum content before submitting to RROB:
\begin{itemize}
	\item Cost Impact --- According to risk table (need to add risk table to RDO-71)
	\item Likelihood --- According to risk table (need to add risk table to RDO-71)
	\item Schedule Impact --- According to risk table (need to add risk table to RDO-71)
	\item Cost or Schedule Impact Description, if over certain severity.
	\item Cost or Schedule fields under Analyze Risk Quantitative, if over certain severity.
	\item Plan Type, and a minimum of one related Reponse Plan
	\item Realized Risk Plan --- Description of event trigger
\end{itemize}

Optional fields:
\begin{itemize}
	\item Overall Impact --- Impact category that can override the automatically generated cost and schedule impact categories (sets to most severe)
	\item Cost or Schedule Impact Description
	\item All fields under Analyze Risk Quantitative --- note that these do not automatically populate any required fields.
	\item Existential Risk --- RROB or POC set?
	\item Related Actions
\end{itemize}

Automatically generated fields:
\begin{itemize}
	\item All fields under Analyze Risk Score.
\end{itemize}

\subsection{Continuous Reivew and Updates to Active Risks}

TBD for RROB monthly meetings.
Managing departments should review risks on a frequency based on severity.

\subsection{Realized Risk Trigger}

Reivew of all Reponse Plans and Actions by POC, and follow-up review with department management.
Department management review with RROB (required if over certain severity).
Review with DO, RROB and department for any escalation, with any review and discussion with NOIRLab.

\subsection{Retiring Risks}

Review of close-out due to realized risk, or if event cannot trigger.
RROB to review before retiring risk that didn't trigger.
All actions should be completed in the Risk Tool (could include Jira tickets for follow-up work) before retiring risk that did trigger.
Documentation updates should be considered.


\section{Risk \& Opportunity Response\, Monitoring\, and Control}

This section describes how risks and opportunities responses, monitored and controlled.

\subsection {Risk \& Opportunity Response}

The definition of responses are included in the ATS Risk Software.
The following include the definition of response plans to risks:

\begin{itemize}
	\item \textbf{Avoid} ---
	changing your strategy or plans to avoid the risk. This risk response strategy is about removing the threat by any means. That can mean changing your management plan to avoid the risk because it’s detrimental to the project/program.

	\item \textbf{Transfer} ---
	passing ownership and/or liability to a third party - Transfer or pass the work on resolving the risk to a third party. For e.g. purchase fire insurance for an unfinished building.

	\item \textbf{Mitigate} ---
	reducing the probability and/or impact of the risk below a threshold of acceptability - Some risks cannot be avoided and need to take action to reduce the impact of the risk. For e.g. work procedures and equipment designed to reduce workplace safety risks.

	\item \textbf{Accept} ---
	recognizing residual risks and devising responses to control and monitor them - This risk response strategy consists of identifying a risk and documenting all the risk management information about it, but not taking any action unless the risk is realized. 
\end{itemize}

The following include the definition of response plans to opportunities.

\begin{itemize}
	\item \textbf{Exploit} ---
	Exploiting a risk to make use of the opportunity that becomes available if that risk occurs.

	\item \textbf{Share} ---
	Distributing the risk across multiple stakeholders (teams/projects/programs)

	\item \textbf{Enhance} ---
	Enhancements is an action that is taken to increase the chance of the opportunity occurring.

	\item \textbf{Ignore} ---
	Opportunities that cannot be actively addressed through exploiting, sharing or enhancing can perhaps be ignored, with no special measures being taken to address them.
\end{itemize}




\newpage

\appendix

% Include all the relevant bib files.
% https://lsst-texmf.lsst.io/lsstdoc.html#bibliographies
\section{References} \label{sec:bib}
\renewcommand{\refname}{} % Suppress default Bibliography section
\bibliography{local,lsst,lsst-dm,refs_ads,refs,books}

% Make sure lsst-texmf/bin/generateAcronyms.py is in your path
\section{Acronyms} \label{sec:acronyms}
\addtocounter{table}{-1}
\begin{longtable}{p{0.145\textwidth}p{0.8\textwidth}}\hline
\textbf{Acronym} & \textbf{Description}  \\\hline

DM & Data Management \\\hline
NOIRLab & NSF's National Optical-Infrared Astronomy Research Laboratory; \url{https://nationalastro.org} \\\hline
RDO & Rubin Directors Office \\\hline
SP & Story Point \\\hline
TBD & To Be Defined (Determined) \\\hline
\end{longtable}

% If you want glossary uncomment below and comment out the two lines above.
% \printglossaries

\end{document}
