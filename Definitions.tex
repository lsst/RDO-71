\subsection{Risk \& Opportunity Definitions}
\label{sec:definitions}

\textbf{Risk} (also known as \textbf{Threat}) ---
 The degree of exposure to an event that might happen to the \emph{detriment}  of a program, project, or other activity. 
 It is described by a combination of the probability that the risk event will occur and the consequence of the extent of loss from the occurrence, or impact.
 Risk is an inherent part of all activities, whether the activity is simple and small, or large and complex
Risks are categorized into four types depending on how they impact operations --- technical, budget (cost), schedule and programmatic.

The term \emph{risk} is used in this document to refer to a risk that has a negative impact, also known as a "threat". 
The term "risk" that has a positive impact is an \emph{opportunity}, and is described below.

\textbf{Issue} ---
A risk that has been realized, e.g. the undesired outcome has materialized.

\textbf{Technical Risk} ---
The possibility that a technical requirement of the system may not be achieved in the system life cycle.
Technical risk exists if the system may fail to achieve performance requirements; to meet operability, producibility, testability, or integration requirements; or to meet environmental protection requirements.
A potential failure to meet any requirement that can be expressed in technical terms is a source of technical risk (INCOSE, pg. 220).

\textbf{Budget (Cost) Risk} ---
The possibility that available budget for operations will be reduced or insufficient to cover operations activity.
Cost risk exists if: a) Rubin Observatory must devote more resources than planned to achieve technical requirements; b) Rubin Observatory must add resources to support slipped schedules due to any reason; c) if changes must be made to the scope of operations; or d) if changes occur in the organization (i.e., Rubin Observatory, Association of Universities for Research in Astronomy [AURA] and/or NOIRLab, SLAC) or national economy.
Budget risk can be predicted at the total operations level or for a system element or activity.
The collective effects of activity or element-level cost risk can produce cost risk for Rubin Observatory generally (INCOSE, pg. 220).

\textbf{Schedule Risk} ---
The possibility that Rubin Observatory will fail to meet scheduled milestones.
Schedule risk exists if there is inadequate allowance for delays in survey execution.
Schedule risk exists if difficulty is experienced in achieving schedule accomplishments, such as the timely accumulation of survey data or data release.
Schedule risk can be incurred at the total operations level for milestones such as deployment of the first data release.
The cascading effects of activity or element-level schedule risks can produce schedule risk for Rubin Observatory generally (INCOSE, pg. 220).

\textbf{Programmatic Risk} ---
Produced by events that are beyond the control of the Rubin Observatory management team.
These events often are produced by decisions made by personnel at higher levels of authority, such as reductions in Rubin Observatory priority, delays in receiving authorization to proceed with a change to the operations plan, reduced or delayed funding, changes in organization or national objectives, etc.
Programmatic risk can be a source of risk in any of the other three risk categories (INCOSE, pg. 220).
AURA holds an independent risk register for Rubin Observatory Operations.

\textbf{Risk Management} ---
The art and science of planning, assessing, and handling future events to avoid unfavorable impacts on Rubin Observatory budget, schedule, or performance to the extent possible.
Risk management is a structured, formal, and disciplined activity focused on the necessary steps and planning actions to determine and control risks to an acceptable level.
Risk Management is an event-based management approach to managing uncertainty.

\textbf{Risk and Opportunity Responses} ---
Responses are strategic process(es) controlling identified risks, whereby the stakeholders decide how to deal with each risk be it opportunities or threats.

\textbf{Risk and Opportunity Actions} ---
Actions taken to implement a response plan if a threat or opportunity is realized.

Described below are the four basic response types to risks (threats).

\textbf{Avoidance} ---
Avoid the risk through the change of requirements or redesign (INCOSE, pg. 223).

\textbf{Mitigation} (also known as \textbf{Control}) ---
Requires the expenditure of budget or other resources to reduce the likelihood and/or consequence.
Mitigations are proposed activities that are not part of the operations baseline.
Once mitigation activities are approved, they are just baselined activities and no longer referred to as mitigations.

\textbf{Transfer} ---
Transferring responsibility for the risk by agreement with another party that it is in their scope to mitigate and respond to impacts if the risk is realized. Purchasing insurance is an example of transferring risk.

\textbf{Acceptance} ---
Accept the risk and the consequences of it becoming realized.

% \textbf{Existential Risk} ---

\textbf{Opportunity} (also known as \textbf{Benefit} ---
The degree of exposure to an event that might happen to the benefit of a program, project, or other activity.
It is described by a combination of the probability that the opportunity event will occur and the consequence of the extent of gain from the occurrence, or impact.
There are two levels of opportunities. At the macro level, a project itself is the manifestation of the pursuit of an opportunity.
At the element level, tactical opportunities exist, whereby certain events, if realized, provide a cost or schedule savings to the project or increase survey performance.

\textbf{Opportunity Management} ---
The proactive art and science of planning, assessing, and handling future events to seek favorable impacts on project, cost, schedule, or performance to the extent possible.
Opportunity management is a structured, formal, and disciplined activity focused on the necessary steps and planning actions to determine and exploit opportunities to the extent possible.

Described below are the four basic response types to opportunities.

\textbf{Exploit} ---
Increase the likelihood of the opportunity occurring through the expenditure of budget or other resources.
The expenditure of budget should be evaluated against the probability weighted exposure of the opportunity to ensure that the likely net payoff is a positive one.

\textbf{Share} ---
Distributing the risk across multiple stakeholders (teams/projects/programs).

\textbf{Enhance} ---
An action that is taken to increase the chance of the opportunity occurring.

\textbf{Ignore} (also known as \textbf{Acceptance}) ---
Accept the opportunity as stated and do nothing about it for the time being.

\textbf{Contingency} ---
Rubin Observatory Operations has no formal "cash" contingency.
Contingency can be gained by reducing scoped activities or deliverables in the operations plan.

\textbf{Contingency Management} ---
The formal process that provides the project the ability and flexibility to solve unforeseen issues that may impact the project’s budget, schedule, and technical performance.
The process incorporates activity-based uncertainties and high impact event-based uncertainties.
This is fundamentally the annual and multiyear planning cycle for operations.
It is expected such replanning could happen on timescales of less than one year for dramatic events (unexpected reduced funding from the NSF and/or DOE).
Regular annual planning cycles may result in modest changes to scope and deliverables based on annual appropriations, and Rubin Operations will work with the agencies to plan on multiyear timescales.

\textbf{Event} ---
A specific incident/item that occurs at unique points in time (specific time, distributed time period, or random) during operations.
Events are defined in terms of something happening and are independent of activities.
Events with negative consequences, when coupled with their probability of occurrence, impact to operations, and handling approach, are the basis for defined entries in the Risk Register.

\textbf{Activity} ---
A specific task or set of tasks that are required by Rubin Observatory Operations, use up resources, and take time to complete (Project Management, pg. 338).

